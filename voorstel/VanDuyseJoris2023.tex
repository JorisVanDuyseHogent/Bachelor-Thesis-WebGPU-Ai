\documentclass{hogent-article}

\addbibresource{references.bib}

\studyprogramme{Professionele bachelor toegepaste informatica}
\course{Bachelors Proef}
\assignmenttype{Paper: Onderzoeksvoorstel}
\academicyear{2023–2024}

\title{Optimalisatie van AI-Integratie in PWA's via WebGPU: Vereenvoudiging van Installatieprocessen en Verbetering van Prestaties op de Client-side}

\author{Joris Van Duyse}
\email{joris.vanduyse@student.hogent.be}

% \projectrepo{https://github.com/JorisVanDuyseHogent/Dissertation-Proposal-WebGPU-Ai}

\specialisation{Mobile \& Enterprise development}
\keywords{
  Web Development,
  WebGPU,
  Progressive Web Apps (PWA's),
  Opensource AI-modellen,
  Client-side uitvoering,
  Webgebaseerde AI,
  Whisper-model,
  AI in de browser,
  Webontwikkeling,
  Gedistribueerde systemen,
  Geavanceerde webtechnologieën,
  User Experience (UX),
  Privacy,
  Schaalbaarheid,
  Webapplicatie-ontwikkeling
}

\begin{document}

\begin{abstract}
  Hier neem je de abstract van je onderzoeksvoorstel over.
\end{abstract}

\tableofcontents

\bigskip

\section{Inleiding}%
\label{sec:inleiding}

% TODO: (fase 1) introduceer je gekozen onderwerp, formuleer de onderzoeksvraag en deelvragen. Wat is de doelstelling (is die S.M.A.R.T.?), wat zal het resultaat zijn van het onderzoek (een Proof-of-Concept, een prototype, een advies, ...)? Waarom is het nuttig om dit onderwerp te onderzoeken?
De snelle evolutie van webtechnologieën heeft de deur geopend naar nieuwe mogelijkheden voor de uitvoering van complexe taken in de browser. 
Binnen het domein van kunstmatige intelligentie (AI) en opensource-modellen, 
biedt dit onderzoek een innovatieve benadering door gebruik te maken van WebGPU voor het lokaal uitvoeren 
van AI-modellen op de browser. Deze aanpak gaat verder dan louter het verplaatsen van rekenkracht van servers naar de client-side; 
het benadrukt ook de essentie van het aanpassen aan de huidige trend van Progressive Web Apps (PWA). 
Tegelijkertijd biedt dit de mogelijkheid om het installatieproces van AI-modellen aanzienlijk te vereenvoudigen.


\section{Literatuurstudie}%
\label{sec:literatuurstudie}

% TODO: (fase 4) schrijf de literatuurstudie uit en gebruik waar gepast referenties naar de vakliteratuur.

% Refereren naar de literatuur kan met:
% \autocite{BIBTEXKEY} -> (Auteur, jaartal)
% \textcite{BIBTEXKEY} -> Auteur (jaartal)
Voorbeeld van een referentie waar de auteursnaam geen onderdeel van de zin is~\autocite{Moore2002}.


\section{Methodologie}%
\label{sec:methodologie}

% TODO: (fase 5) beschrijf in detail in welke fasen je onderzoek uiteenvalt, hoe lang elke fase duurt en wat het concrete resultaat van elke fase is. Welke onderzoekstechniek ga je toepassen om elk van je onderzoeksvragen te beantwoorden? Gebruik je hiervoor experimenten, vragenlijsten, simulaties? Je beschrijft ook al welke tools je denkt hiervoor te gebruiken of te ontwikkelen.


\section{Verwachte resultaten}%
\label{sec:verwachte-resultaten}

% TODO: (fase 6) beschrijf wat je verwacht uit je onderzoek en waarom (bv. volgens je literatuuronderzoek is softwarepakket A het meest gebruikte en denk je dat het voor deze casus ook het meest geschikt zal zijn). Natuurlijk kan je niet in de toekomst kijken en mag je geen alternatieve mogelijkheden uitsluiten. In de praktijk gebeurt het ook vaak dat een onderzoek tot verrassende resultaten leidt, dat maakt het proces nog interessanter!
De implementatie van complexe AI-modellen, zoals Whisper AI en Midjourney, 
in een webomgeving met behulp van WebGPU zal ongetwijfeld een uitdagend proces zijn. 
Het streven naar succes in dit onderzoek manifesteert zich niet alleen in de succesvolle integratie van deze modellen in Progressive Web Apps (PWA's), 
maar ook in het bereiken van prestaties op het niveau van gevestigde systemen, zoals Whisper AI met CUDA. 
De complexiteit van deze taak wordt benadrukt door de noodzaak om de rekenkracht van WebGPU te optimaliseren, 
waarbij het doel is om vergelijkbare prestaties te behalen als die welke worden geboden door meer traditionele uitvoeringsomgevingen. 
Het succes van dit onderzoek zal niet alleen worden afgemeten aan de volledige en stabiele werking van de geïmplementeerde AI-modellen op de client-side, 
maar ook aan de mate waarin WebGPU een gelijkwaardige of zelfs verbeterde prestatie kan leveren in vergelijking met CUDA. 
Het streven naar deze prestatie-equivalentie met gevestigde technologieën markeert een significante mijlpaal 
en draagt bij aan het begrip van de mogelijkheden van WebGPU voor het uitvoeren van veeleisende AI-taken binnen webomgevingen.

\section{Discussie, conclusie}%
\label{sec:discussie-conclusie}


%------------------------------------------------------------------------------
% Referentielijst
%------------------------------------------------------------------------------
% TODO: (fase 4) de gerefereerde werken moeten in BibTeX-bestand
% bibliografie.bib voorkomen. Gebruik JabRef om je bibliografie bij te
% houden.

\printbibliography[heading=bibintoc]

\end{document}