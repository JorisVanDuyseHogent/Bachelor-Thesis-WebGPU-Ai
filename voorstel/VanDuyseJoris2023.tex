\documentclass{hogent-article}
\addbibresource{references.bib}


\studyprogramme{Professionele bachelor toegepaste informatica}
\course{Bachelorproef}
\assignmenttype{Paper: Onderzoeksvoorstel}
\academicyear{2023–2024}

\title{Optimalisatie van AI-Integratie in PWA's via WebGPU: Vereenvoudiging van Installatieprocessen en Verbetering van Prestaties op de Client-side}

\author{Joris Van Duyse}
\email{joris.vanduyse@student.hogent.be}

\projectrepo{https://github.com/JorisVanDuyseHogent/Dissertation-Proposal-WebGPU-Ai}

\specialisation{Mobile \& Enterprise development}
\keywords{
  Web Development,
  WebGPU,
  Progressive Web Apps (PWA's),
  Opensource AI-modellen,
  Client-side uitvoering,
  Webgebaseerde AI,
  Whisper-model,
  AI in de browser,
  Webontwikkeling,
  Gedistribueerde systemen,
  Geavanceerde webtechnologieën,
  User Experience (UX),
  Privacy,
  Schaalbaarheid,
  Webapplicatie-ontwikkeling
}

\begin{document}

\begin{abstract}
\end{abstract}

\tableofcontents

\bigskip

\section{Inleiding}%
\label{sec:inleiding}

% TODO: (fase 1) introduceer je gekozen onderwerp, formuleer de onderzoeksvraag en deelvragen. Wat is de doelstelling (is die S.M.A.R.T.?), 
% wat zal het resultaat zijn van het onderzoek (een Proof-of-Concept, een prototype, een advies, ...)? Waarom is het nuttig om dit onderwerp te onderzoeken?

Dit onderzoek verkent de grenzen van webtechnologieën door zich te richten op de integratie van opensource kunstmatige intelligentie (AI) modellen 
met WebGPU in Progressive Web Apps (PWAs). 
De snelle evolutie van webtechnologieën, zoals beschreven door \textcite{Shumylo2023}, 
biedt nieuwe mogelijkheden voor het uitvoeren van complexe taken rechtstreeks in de browser.
In het specifieke domein van AI en opensource-modellen presenteert dit onderzoek een innovatieve benadering 
waarbij WebGPU wordt ingezet voor de lokale uitvoering van AI-modellen in de browser.

\bigbreak

Deze benadering gaat verder dan de traditionele verschuiving van rekenkracht van servers naar de client-side. 
Het legt de nadruk op de adaptatie aan de hedendaagse trend van Progressive Web Apps (PWA), 
waarmee de gebruikerservaring aanzienlijk kan worden verbeterd door de installatie van AI-modellen te vereenvoudigen. 
Na het opzetten van de proof-of-concept zal een vergelijking plaatsvinden op het gebied van performantie tussen WebGPU en een uitvoering van een model met CUDA, 
waarmee inzichten worden verkregen in de efficiëntie van deze benaderingen.

\bigbreak

Tevens richt dit onderzoek zich op de gebruiksvriendelijkheid van opensource AI-modellen binnen PWAs, 
waarbij de combinatie van PWA en WebGPU wordt onderzocht om het installatieproces van deze modellen te stroomlijnen en toegankelijk te maken voor een breder publiek. 
De studie omvat ook een grondige analyse van veiligheidsaspecten, waarbij de verschillen tussen WebGL en WebGPU met betrekking tot de toegang tot hardware grafische kaarten worden onderzocht. 
Deze inzichten dragen bij aan het begrip van zowel de mogelijkheden als de beperkingen van geavanceerde webtechnologieën in het domein van AI-integratie.

\section{Literatuurstudie}%
\label{sec:literatuurstudie}

Als onderdeel van mijn onderzoek omtrent de integratie van opensource kunstmatige intelligentie (AI) 
modellen met WebGPU binnen Progressive Web Apps (PWAs), 
wordt de aanzienlijke waarde van PWAs als een moderne paradigmaverschuiving binnen webapplicaties benadrukt. 
De PWAs, zoals gepresenteerd door \textcite{Shumylo2023}, 
zijn zorgvuldig vormgegeven om een ervaring te bieden die vergelijkbaar is met native apps, 
gebruikmakend van geavanceerde webtechnologieën zoals Service Workers, Web App Manifest, en Push Notifications.

Deze technologische pijlers empoweren PWAs om functionaliteiten zoals offline-ondersteuning, 
snelle laadtijden, en pushmeldingen te integreren, 
waarmee zij een aantrekkelijke keuze worden voor gebruikers die streven naar een naadloze en betrokken mobiele interactie.

In tegenstelling tot traditionele webapplicaties vertoont het PWA-concept een reeks significante voordelen, 
waaronder versnelde laadtijden, 
robuuste offline-ondersteuning, een ervaring die parallel loopt aan native apps, 
grensoverschrijdende compatibiliteit, en moeiteloze ontdekbaarheid. 
Deze eigenschappen maken PWAs bijzonder geschikt als omgeving voor de uitvoering van complexe AI-taken aan de client-zijde, 
met de ondersteuning van WebGPU.

Dit onderzoek richt zich op een grondige verkenning van de mogelijkheden om opensource AI-modellen te integreren in deze vooruitstrevende webtoepassingen, 
met als doel de algehele gebruikerservaring te versterken en innovatieve inzichten te verschaffen binnen het domein van Mobile en Enterprise Development.


% Refereren naar de literatuur kan met:
% \autocite{BIBTEXKEY} -> (Auteur, jaartal)
% \textcite{BIBTEXKEY} -> Auteur (jaartal)


\section{Methodologie}%
\label{sec:methodologie}

% TODO: (fase 5) beschrijf in detail in welke fasen je onderzoek uiteenvalt, hoe lang elke fase duurt en wat het concrete resultaat van elke fase is. Welke onderzoekstechniek ga je toepassen om elk van je onderzoeksvragen te beantwoorden? Gebruik je hiervoor experimenten, vragenlijsten, simulaties? Je beschrijft ook al welke tools je denkt hiervoor te gebruiken of te ontwikkelen.

\subsection*{Fase 1: WebGL en WebGPU}

\begin{itemize}
\item \textbf{Doelstelling}: Onderzoek naar de vershillen tussen WebGL en WebGPU
\item \textbf{Aanpak}:
\begin{itemize}
\item Wat bracht WebGL en wat ontbreekt het.
\item Wat is WebGPU en wat zijn de voordelen.
\item Hoe kan WebGPU worden gebruikt in een PWA.
\end{itemize}
\item \textbf{Tijdskader}: 2 weken
\item \textbf{Deliverable}: Een overzicht van de verschillen tussen WebGL en WebGPU.
\end{itemize}

Er wordt onderzocht in hoever WebGPU een verbetering is ten opzichte van WebGL. De performantie van beide technologieën wordt vergeleken. 
Er wordt ook onderzocht welke technologieën er nodig zijn om WebGPU te gebruiken en hoe deze verschillen met WebGL.

2 Weken

\subsection*{Fase 2: WebGPU, privacy en veiligheid}

\begin{itemize}
\item \textbf{Doelstelling}: Veiligheidsrisico analyse van WebGPU
\item \textbf{Aanpak}:
\begin{itemize}
\item 
\item 
\item 
\end{itemize}
\item \textbf{Tijdskader}: 2 weken
\item \textbf{Deliverable}: Een analyse van de veiligheidsrisico's van WebGPU.
\end{itemize}

\subsection*{Fase 3: Proof of concept, Ai en WebGPU}

\begin{itemize}
\item \textbf{Doelstelling}: Ai modellen integreren in een PWA met behulp van WebGPU
\item \textbf{Aanpak}:
\begin{itemize}
\item Een geschikt Ai model zoeken.
\item Een simpele PWA opzetten.
\item WebGPU integreren in de PWA.
\end{itemize}
\item \textbf{Tijdskader}: 6 weken
\item \textbf{Deliverable}: Een proof of concept van een Ai model in een PWA met behulp van WebGPU.
\end{itemize}
Er wordt onderzocht in hoeverre het Whisper AI model kan geïmplementeerd worden in een PWA met behulp van WebGPU.

\subsection*{Fase 4: Performantie WebGPU}

\begin{itemize}
\item \textbf{Doelstelling}: Performantie van WebGPU vergelijken met CUDA
\item \textbf{Aanpak}:
\begin{itemize}
\item Proof of concept testen aan de hand van varschillende benchmark.
\item Een CUDA implementatie van het Whisper AI model opzetten.
\item Vergelijken van de performantie van beide implementaties.
\end{itemize}
\item \textbf{Tijdskader}: 2 weken
\item \textbf{Deliverable}: Een vergelijking van de performantie van WebGPU met CUDA.
\end{itemize}

Na het uitwerken van een proof of concept kan de performantie van WebGPU vergeleken worden met alternatieve technologieën zoals CUDA.
Er zijn namelijk meerdere AI-modellen die gebruik maken van CUDA, zoals Whisper AI van OpenAI.

\section{Verwachte resultaten}
\label{sec:verwachte-resultaten}

% TODO: (fase 6) beschrijf wat je verwacht uit je onderzoek en waarom (bv. volgens je literatuuronderzoek is softwarepakket A het meest gebruikte en denk je dat het voor deze casus ook het meest geschikt zal zijn). Natuurlijk kan je niet in de toekomst kijken en mag je geen alternatieve mogelijkheden uitsluiten. In de praktijk gebeurt het ook vaak dat een onderzoek tot verrassende resultaten leidt, dat maakt het proces nog interessanter!
De implementatie van complexe AI-modellen, zoals Whisper AI en Midjourney, 
in een webomgeving met behulp van WebGPU zal ongetwijfeld een uitdagend proces zijn. 
Het streven naar succes in dit onderzoek manifesteert zich niet alleen in de succesvolle integratie van deze modellen in Progressive Web Apps (PWA's), 
maar ook in het bereiken van prestaties op het niveau van gevestigde systemen, zoals Whisper AI met CUDA. 
De complexiteit van deze taak wordt benadrukt door de noodzaak om de rekenkracht van WebGPU te optimaliseren, 
waarbij het doel is om vergelijkbare prestaties te behalen als die welke worden geboden door meer traditionele uitvoeringsomgevingen. 
Het succes van dit onderzoek zal niet alleen worden afgemeten aan de volledige en stabiele werking van de geïmplementeerde AI-modellen op de client-side, 
maar ook aan de mate waarin WebGPU een gelijkwaardige of zelfs verbeterde prestatie kan leveren in vergelijking met CUDA. 
Het streven naar deze prestatie-equivalentie met gevestigde technologieën markeert een significante mijlpaal 
en draagt bij aan het begrip van de mogelijkheden van WebGPU voor het uitvoeren van veeleisende AI-taken binnen webomgevingen.

\section{Discussie, conclusie}

\label{sec:discussie-conclusie}


%------------------------------------------------------------------------------
% Referentielijst
%------------------------------------------------------------------------------
% TODO: (fase 4) de gerefereerde werken moeten in BibTeX-bestand
% bibliografie.bib voorkomen. Gebruik JabRef om je bibliografie bij te
% houden.

% \printbibliography[heading=bibintoc]
\printbibliography

\end{document}